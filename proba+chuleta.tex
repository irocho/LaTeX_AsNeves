%\documentclass[addpoints,12pt]{exam}
\documentclass[12pt]{exam}

\usepackage[spanish,galician]{babel}
\usepackage[utf8]{inputenc}
\usepackage{graphicx}
\usepackage{fancybox}
\usepackage{minibox}
\usepackage{wrapfig}
%--------------------------
%   Interlineado
%--------------------------
\renewcommand{\baselinestretch}{2}
\linespread{1,25}

\begin{document}
\pointname{ punto} % traducido

%\begin{figure}[ht]
%	\raggedright
%	\includegraphics[width=4cm]{i-rocho.png}
%\end{figure}
%
%--------------------------
%   Nome do alumno
%--------------------------
\makebox[\textwidth]{Nome:\enspace\hrulefill}
\vspace{.5cm}

%--------------------------
%   Caixiña título
%--------------------------
\begin{center}
\fbox{\fbox{\parbox{5.5in}{\centering
Electricidade
}}}
\end{center}
\vspace{.5cm}

%--------------------------
%   Preguntas
%--------------------------

\begin{questions}
\question[1] 
Explica
\question[1] 
Explicapolímetro.
\question[1] 
 Propónapagada.
 
 \question[1]  
 Cal$2,2 mA$?
     
 \question[1]       
importancia
\question  
Unircuíto 
 \begin{parts}
	\part[1]
		Representa 
	\part[1]
		Calcula
	\part[1]	
		resistencia 
\end{parts}     
 \end{questions}
 
 
\vspace{.75cm}
\centering

\shadowbox{%
\begin{minipage}[c]{5cm}
\sf \small
Chuleta:
$$R_{eq}=R_1+R_2$$ 
$$R_{eq}=\frac{R_1\times R_2}{R_1+R_2}$$
$$ V=I\times R $$ 
$$ P=V\times I$$
$$ E_{calor}=R\times I^{2} \times t $$
\end{minipage}}



 
%\shadowbox{sombra3d negrita}
%\Ovalbox{esquinitas redondeadas}
           
           
% \makeemptybox{2cm}
% \fillwithlines{2cm}
% \fillwithdottedlines{3cm}        
% \textit{cd} })

%
%\begin{figure}[h!]
%\begin{center}
%\includegraphics[width=11cm]{tecnico.png}
%\caption{  Datos dous ventiladores }
%\end{center}
%\end{figure}



\end{document}