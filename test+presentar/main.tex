%---------------------------------------------------------------------------------------------
 %  Contido :
 %        diapos con:  \begin{frame}..... \end{frame}
 % 	    texto que vai saír por impresora: fóra dos frame
 %        notas íntimas que podo imprimir á parte \note{}
 %---------------------------------------------------------------------------------------------
 
% Lembrar non \documentclass.... 

%---------------------------------------------------------------------------------------------
 %  Paquetería común 
  %---------------------------------------------------------------------------------------------
\usepackage[spanish,galician]{babel}
\usepackage[utf8]{inputenc}
\usepackage{graphicx}
%\usepackage{epsfig} 	% for figures
%\usepackage{xcolor} 	% for color
%\usepackage{amssymb,amsmath}


%   Simple ou dobre columna
 \usepackage{multicol}

%ligazóns
\usepackage{hyperref}	

%----------------------------------------------------------------------------------------------------------------
%  Títulos
%---------------------------------------------------------------------------------------------------------------- 
		
\newcommand{\titulito}{
Profundo momento da vida
}
\subtitle{
Podo no fondo facer un pequeno resumo 
}


\newcommand{\asignatura}{
Sistemas Operativos Monoposto
%Montaxe e Mantemento
}

\begin{document}
%----------------------------------------------------------------------------------------------------------------
%  Títulos segundo o tipo de documento
%---------------------------------------------------------------------------------------------------------------- 
     
\begin{frame}
\mode <beamer>{
\maketitle
}
\end{frame}

% o título queda máis chulo sen ser a dúas columnas
\mode <article>{
\begin{multicols}{1}
\maketitle
\end{multicols}
}

%----------------------------------------------------------------------------------------------------------------
% Contido
%---------------------------------------------------------------------------------------------------------------- 
\abstract{
	Como non está nun frame non se verá nas presentacións
	}
\note{
	Só cando compile anotacións verei esto
	}

Son a primeira liña despois do resumo da columna da esquerda e saio no folio que vou imprimir
\begin{frame}{Linus Torvalds \dots}

\begin{itemize}
\item programou sistemas operativos partindo de cero \pause
\item morreu nos anos 70 \pause
\item vivíu no século XIX \pause
\item creou un sistema operativo de código aberto\pause
\end{itemize}
\end{frame}

\note{vai indo fora}
Son a  liña da columna da esquerda que vai xusto despois da primera diapo e saio no folio que vou imprimir

\begin{frame}{Un exemplo de distro de Linux \dots}
\begin{enumerate}
\item Debian \pause
\item Slackware \pause
\item Red Hat Enterprise \pause
\item todas as respostas son correctas\pause
\end{enumerate} 
\end{frame}
\note{despois das distros}


\begin{frame}{Un exemplo de distro de Linux \dots}
\begin{enumerate}
\item Debian \pause
\item Slackware \pause
\item Red Hat Enterprise \pause
\item todas as respostas son correctas\pause
\end{enumerate} 
\end{frame}
\note{despois das distros}

O epílogo escribiráse
O epílogo escribiráseO epílogo escribiráseO epílogo escribiráseO epílogo escribiráseO epílogo escribiráseO epílogo escribiráseO epílogo escribiráse
%%%%%%

\end{document}

\note{       }

%\begin{figure}
%\includegraphics{./debuxos/micro.png}
%\end{figure}



\begin{frame}{Un exemplo de distro de Linux \dots}
\begin{enumerate}
\item  \pause
\item \pause
\item  \pause
\item        \pause
\end{enumerate} 
\end{frame}














