%---------------------------------------------------------------------------------------------
 %  Contido :
 %        diapos con:  \begin{diapo}..... \end{diapo}
 % 	    texto que vai saír por impresora: fóra dos frame
 %        notas íntimas que podo imprimir á parte \note{}
 %---------------------------------------------------------------------------------------------
 
% Lembrar non \documentclass.... 

%----------------------------------------------------------------------------------------------------------------
%  Títulos
%---------------------------------------------------------------------------------------------------------------- 
\newcommand{\titulito}{
Caracterización de Sistemas Operativos
}
%\subtitle{
%Cuestións sen resposta en contexto
%}
\newcommand{\asignatura}{
Sistemas Operativos Monoposto
%Montaxe e Mantemento e sistemas
}



\begin{document}

%%----------------------------------------------------------------------------------------------------------------
%%  Títulos segundo o tipo de documento
%%---------------------------------------------------------------------------------------------------------------- 
\mode <beamer>{
	\begin{frame}
		\maketitle
	\end{frame}
}
\mode <article>{
	\maketitle
}

%%----------------------------------------------------------------------------------------------------------------
%% Contido
%%---------------------------------------------------------------------------------------------------------------- 
%\abstract{			}%sae no main. article
%
sfhak 
 a
 df adf ds
\begin{diapo} \begin{frame}{ operativo   \dots} 
\begin{enumerate}
	\item hardware\pause
	\item software \pause
	\item malware 
\end{enumerate} \end{frame}  \end{diapo}  
%parella
\begin{diapo}\begin{frame}{ hardware   \dots}
\begin{enumerate}
	\item drivers \pause
	\item terminais \pause
	\item sistemas 
\end{enumerate} \end{frame} \end{diapo}
asdf kkd ad
f asd
%
%%---------------------------------------------------------------------------------------------
%%  Referencias e índice 
%%---------------------------------------------------------------------------------------------
%%\clearpage
%%\begin{thebibliography}{1}
%%	\bibitem{sanclemente} Apuntes do Instituto San Clemente  https://manuais.iessanclemente.net/index.php/Introdución_aos_Sistemas_Operativos
%%	\bibitem {RaMa}
%%	Laura Raya, Miguel A. Martínez, Sistemas Operativos Monopuesto. Editorial RaMa
%%\end{thebibliography}
%%\tableofcontents


\end{document}

%\note{       }

%\begin{figure}
%\includegraphics{./debuxos/micro.png}
%\end{figure}
%
%\begin{diapo} \begin{frame}{ operativo   \dots} 
%\begin{enumerate}
%	\item hardware\pause
%	\item software \pause
%	\item malware 
%\end{enumerate} \end{frame}  \end{diapo}  
%%parella
%\begin{diapo}\begin{frame}{ hardware   \dots}
%\begin{enumerate}
%	\item drivers \pause
%	\item terminais \pause
%	\item sistemas 
%\end{enumerate} \end{frame} \end{diapo}
%
